% #! /usr/bin/pdflatex \catcode35=14 \input
% \catcode`\#=6
%%%%%%%%%%%%%%%%%%%%%%%%%%%%%%%%%%%%%%%%%%%%
% for this unique shebang, read more here:
%  https://tex.stackexchange.com/a/216002
%  the correct solution here is /usr/bin/env -[v]S pdflatex ...
%  but it hangs, and I cannot debug it [easily], just replace
%  the absolute path (run `which pdflatex`) w/ your own
%%%%%%%%%%%%%%%%%%%%%%%%%%%%%%%%%%%%%%%%%%%%
% nice guide if we forget something somewhat inconsequential
% https://www.overleaf.com/learn/latex/Learn_LaTeX_in_30_minutes#Including_title,_author_and_date_information
%%%%%%%%%%%%%%%%%%%%%%%%%%%%%%%%%%%%%%%%%%%%
% install `texlive' to use
% Use pdflatex to compile this
% DON'T use spaces for filenames
% run "pdflatex -halt-on-error la.tex"
%  "-parse-first-line If the first line of the main input file begins with %& parse it to look for a dump name or a -translate-file option."
%  https://tex.stackexchange.com/a/172818
% The first line of an input file starting with %& 
% is a special directive used in the TeX typesetting system. 
% This line is called the format line, and it is used to 
% specify a precompiled format file that TeX should use 
% for the document.
% A format is a binary file that TeX can load into memory 
% to process a document faster. It contains the memory dump of a 
% TeX engine after it has processed a set of macro definitions. 
% By using a format, TeX can skip the time-consuming step of 
% reading and processing macro definitions every time it 
% processes a document.
% The %& directive tells TeX to use the specified format 
% instead of the default one. For example, if the first 
% line of your document is %&latex, then TeX will use the 
% latex.fmt format file.
%  - Phind (GPT4 Model)
%%%%%%%%%%%%%%%%%%%%%%%%%%%%%%%%%%%%%%%%%%%
% BASIC STRUCTURE
%\documentclass[options]{class}
%Preamble commands, if needed
%
%\begin{document}
%Document text & commands here
%
%\begin{environment1}
%Environment 1 commands & text here

%\end{environment1}
%
%More document text & commands here
%
%\begin{environment2}
%Environment 2 commands \& text here
%\end{environment2}
%
%\end{document} 

% Notes:
% - Document classes: article, letter, report, slides, book, beamer, exam (etc.)
% - Document options: 10pt, 11pt, 12pt; letterpaper, legalpaper, executivepaper, a4paper, a5paper, b5paper; onecolumn, twocolumn; notitlepage; titlepage
% - options for formatting command \documentclass are optional
% - Useful environments: abstract, equation, eqnarray, figure, table, tabular, center, itemize, enumerate, appendix, thebibliography
% - Environments can be nested
% - Comment lines TEX begin with a % . Any input past a % is ignored.

% Common constructs

% Creating a title page
% - \title{Title text}
% - \author{Author names and addresses}
% - the command \maketitle will create the page using the information provided

% Creating logical sections of text: 
% - \part 
% - \chapter 
% - \section 
% - \subsection 
% - \subsubsection
% - \paragraph 
% - \tableofcontents 
% - \listoffigures 
% - \listoftables 

% Mathematics
% - In-line: $y = 3x^2 + 4x + 2$ ($ puts TeX in math mode; second $ exits math mode)
% - Displayed: use \begin{equation} ... \end{equation}
% - Remember to use $a$ (inline) for variable $a$ (e.g.)
% - Common function names are not set in italics, so in math mode use, e.g.: \sin \cos \exp

% Tables
% - For math, use eqnarray env
% - For text, use tabular env
% - Basic syntax
% - \begin{env}{lcr} % env = "eqnarray" or "tabular"
%   a & b & c \\
%   d & e & f
%   \end{env}
% - % where l ==> left, c ==> center, r ==> right, and the number of
% - % these in the argument is the number of columns 
% Captions: \caption{text} (in figure or table env) 

% Cross-referencing: \label{tag} associates a counter with tag
% - In equation environment, counter is equation counter
% - In figure caption, figure counter
% - In table caption, table counter
% - In running text, section counter
% - \ref{tag} references the counter associated with tag
% - Convention: tag $\to$ tag_type:tag_name (e.g.: sec:intro) (usual types: eq sec tbl fig)

% https://inside.mines.edu/~dmwood/fs02-computing/FieldSession02/LaTeX/node8.html
% https://www.overleaf.com/learn/latex/How_to_Write_a_Thesis_in_LaTeX_(Part_5)%3A_Customising_Your_Title_Page_and_Abstract
%%%%%%%%%%%%%%%%%%%%%%%%%%%%%%%%%%%%%%%%%%%%%%%

\documentclass[draft]{article}

\PackageInfo{MyInfoPackageTest}{Testing info log}
\PackageWarningNoLine{MyWarningPackageTest}{Testing warn log}
% \PackageError{MyErrorPackageTest}{error text}{help text}
%  this crashes the compilation

% latex package to check/use draft thingy.
\usepackage{ifdraft}
% good package for setting margins etc.
%\usepackage{fullpage}

% good package for monospace output, e.g. textual output
\usepackage{verbatim}

% minteed is a package for code, uses pygments as a backend
% pygments REQUIRED! texlive-minted* is the fedora package
% this also requires -shell-escape . . .
\usepackage{fancyvrb}
% THIS IS A HACK, READ MORE BELOW
\makeatletter 
\def\KV@FV@lastline@default{%
  \let\FancyVerbStopNum\m@ne
  \let\FancyVerbStopString\relax}
\fvset{lastline}
% THIS IS A HACK, READ MORE BELOW
% authors note: it still didn't work :( guess no colours for us
\makeatother
% https://tex.stackexchange.com/a/623823
\usepackage{minted}
\usemintedstyle{default}
%
% What do \makeatletter and \makeatother do?
%
% All characters in TeX are assigned a Category Code or 
% "catcode". There are 16 catcodes in all, some containing 
% just a single character, e.g. \ is (normally) catcode 0, 
% {, catcode 1 etc. Normal characters are catcode 11; 
% this category normally comprises all of the letter characters. 
% The @ symbol is given the catcode of 12, which means it is 
% not treated as a normal letter. The effects of this are that 
% @ cannot normally be used in user document files as part 
% of a multicharacter macro name. (All other non-letter 
% characters are also forbidden in macro names: for example, 
% \foo123, and \foo?! are not valid macro names.)
%
% In LaTeX class and package files, however, @ 
% is treated as a normal letter (catcode 11) and 
% this allows package writers to make macro-names with 
% @. The advantage of this is that such macro names are 
% automatically protected from regular users: since they 
% cannot use @ as a normal letter, there is no accidental 
% way for a user to override or change a macro that is 
% part of the internal workings of a package.
%
% However, it is sometimes necessary in user documents 
% to have access to such package-internal macros, 
% and so the commands \makeatletter and \makeatother 
% change the catcode of @ from 12 to 11 and 11 to 12, 
% respectively.
% - https://tex.stackexchange.com/a/8353

\title{Hello, Title!}
\author{Key, Reference, 1023, South Dakota}
\date{1821, 1st July}

% Some classes require \title, \author and similar 
% declarations to be placed after \begin{document}.
% IIRC, you won't be able to use babel shorthands 
% if your \title etc. are in the preamble unless 
% you use external packages to enable it.

% allow only \chapter in ToC
\setcounter{tocdepth}{0}
% every \begin needs to close with \end
\begin{document}

\maketitle

\pagebreak

\tableofcontents

\pagebreak

\ifoptiondraft{
    \section{Draft OPTION Section}
    Draft OPTION SET IN DOCUMENT CLASS!
}{
    \section{Final OPTION Section}
    Final OPTION SET IN DOCUMENT CLASS!
}

\ifdraft {
    \section[12pt]{Actual Draft Section}
    This section will only be displayed in draft mode.
}{
    \section{Actual Final Section}
    This section will only be displayed in final mode.
}

\section*{Test section}

% the last \ escapes the space from what I can gather, otherwise it gets stuck together
Hello World! This is my first \LaTeX\ document

One-line code formatting also works with \texttt{minted}. For example, a small fragment of HTML like this:
\mint{html}|<h2>Something <b>here</b></h2>|
\noindent can be formatted correctly.

\begin{verbatim}
// Verbatim is better for textual output, rather than code.
// Hello, world!
class HelloWorld {
    public static void main(String args[]) {
        System.out.println("Hello Java");  
    }
}
\end{verbatim}

% minted code section
% \begin{minted}[linenos, mathescape]{python3}
% # Writing boilerplate code to avoid writing boilerplate code!
% # https://stackoverflow.com/questions/32910096/is-there-a-way-to-auto-generate-a-str-implementation-in-python
%
% # Defined as $\pi=\lim_{n\to\infty}\frac{P_n}{d}$ where $P$ is the perimeter
% # of an $n$-sided regular polygon circumscribing a
% # circle of diameter $d$.
%
% if __name__ == "__main__":
%     print("test??!")
%
% def auto_str(cls):
%     """Automatically implements __str__ for any class."""
%
%     def __str__(self):
%         return '%s(%s)' % (
%             type(self).__name__,
%             ', '.join('%s=%s' % item for item in vars(self).items())
%         )
%
%     cls.__str__ = __str__
%     return cls
% \end{minted}

\hrulefill % or just hrule


\paragraph*{
    This is some example text\footnote{\label{myfootnote}Hello footnote}.
}


% 0 to 4 denotes the pagebreak request "intensity"
% don't use a number to always break page
\pagebreak[4]
% import and embed file
% \inputminted{}{jq.sh}
\section*{My Section}

% generally shouldn’t use \begin{center} … \end{center} 
% for headings. It’s really better used to make lists or 
% similar. A better way to centre your title headings is 
% to use the titlesec package
\paragraph*{
    First section paragraph. boo boo ha ha.
}

\subsection{
    my subsection title
}
\paragraph*{
    subsection paragraph
}
\subsubsection*{
    my subsubsection title
}

\subsection*{
    asterisk affects numbering
}

\paragraph*{
    subsubsection paragraph
}

% The article class doesn't define the \chapter command, 
% because articles don’t have chapters. Instead they have
% sections, subsections, etc. If you really want to use 
% article, then you might want to consider using \section, 
% \subsection and so on.
% 
% If you really want to use the \chapter command, then 
% I’d suggest using the book or report document classes.
%
% \chapter*{
%     Chapter
% }
%
% \chapter{
%     Numbered Chapter
% }
% https://tex.stackexchange.com/a/117163

% block math mode
$$2\left(\frac{1}{x^2-1}\right)$$

% lists
\begin{enumerate}
    \item pencil
    \item calculator
    \item assessments
\end{enumerate}

\begin{itemize}
    \item one item
    \item another
\end{itemize}

% tables
% The “{r|cl}” after the tabular \begin{tabular} indicate the alignment of the three columns: right,
% center, and left. This is mandatory as it specifies the layout of the table. For more columns, type more
% alignment commands, e.g. for a table with 5 columns all aligned to the right, you would use rrrrr.
%  The vertical bar | between the r and c indicates that a vertical line should be drawn between those columns.’
%  The & separates the columns in the body of the table.
%  A \\ signifies the end of each line of the table.
%  The command \hline means that a horizontal line should be inserted
\begin{tabular}{r|cl}
    1st column & 2nd column & 3rd column\\
    \hline
    a & b & c
\end{tabular}

\end{document}

